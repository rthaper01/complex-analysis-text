\chapter{The Exponential and Trigonometric Functions}
\label{chap:exp-trig}
The reader probably recalls the familiar Taylor series expansions for the exponential and trigonometric functions. With complex numbers, these representations become even more powerful, and easier to prove.

Here, we undertake the formal development of the constants $e$ and $\pi$; complex analysis provides an easy way of defining them and appreciating their intimate relationship to each other.

\section{The Exponential}
\begin{definition}
	The \emph{exponential function} is defined as the solution to the differential equation $$f'(z)=f(z)$$ with initial value $f(0)=1$. 
\end{definition}
	
We solve it by setting
\begin{align*}
	f(z) &=a_0+a_1z+\cdots+a_nz^n+\cdots, \\
	f'(z) &=a_1+2a_2z+\cdots+na_nz^{n-1}+\cdots.
\end{align*}
Setting these power series into the differential equation yields $a_{n-1}=na_n$, and the initial condition gives $a_0=1$. It follows by induction that $a_n=1/n!$. We denote this function by $e^z$ (the exponential form will be justified below), so that $$e^z=\sum_{n=0}^{\infty}\dfrac{z^n}{n!}.$$ We also note that this series converges on the whole complex plane, as $\sqrt[n]{n}^{1/n} \rightarrow \infty$. To see why the terms of the series are unbounded, observe that $$n!=n \cdot (n-1) \cdots n/2 \cdots 1 \ge \left(\dfrac{n}{2}\right)^{n/2},$$ so $$n!^{1/n}=\sqrt{\dfrac{n}{2}}.$$ Since the right-hand side clearly grows without bound, so does the left.

It is a consequence of the differential equation that $e^z$ satisfies the identity $$e^{a+b}=e^a \cdot e^b,$$ for $$D(e^z \cdot e^{c-z})=e^z \cdot e^{c-z}+e^z \cdot (-e^{c-z})=0.$$ Hence, $e^z \cdot e^{c-z}$ is constant. The value of the constant is found by setting $z=0$, giving $e^z \cdot e^{c-z}=e^c$, resulting in the additive identity for the exponential.

As a word of caution, we were able to use the fact that a zero derivative implies a constant function because $f$ is analytic on the whole plane. Since $\CC$ is connected, so is $f(\CC)$, and a zero derivative on a connected space implies that the function is constant on that space.

Note that $e^z \cdot e^{-z}=1$, so $e^z$ is never zero. For real $x$ the series expansion shows that $e^x>1$ for $x>0$, and since $e^x$ and $e^{-x}$ are reciprocals, $0<e^x<1$ for $x<0$. The fact that the series has real coefficients shows that $e^{\overline{z}}$ is the conjugate of $e^z$. Hence, $\abs{e^{iy}}^2=e^{iy} \cdot e^{-iy}=1$, and $\abs{e^{x+iy}}=e^x$.

\section{The Trigonometric Functions}
The trigonometric functions are defined by $$\cos z=\dfrac{e^{iz}+e^{-iz}}{2}, \quad \sin z=\dfrac{e^{iz}-e^{-iz}}{2i}.$$ Substituting the power series representation of $e^z$ yields
\begin{align*}
	\cos z &=\sum_{n=0}^{\infty} \dfrac{(-1)^n \cdot z^{2n}}{(2n)!}, \\
	\sin z &=\sum_{n=0}^{\infty} \dfrac{(-1)^n \cdot z^{2n+1}}{(2n+1)!}.
\end{align*}
For real $z$ they reduce to the familiar Taylor developments of $\cos x$ and $\sin x$, with the significant difference that we have now redefined these functions without use of geometry.

Now, Euler's formula becomes apparent: $$e^{iz}=\cos z+i \sin z,$$ as well as the identity $$\cos^2 z+\sin^2 z=1.$$ It follows likewise that $$(\cos z)'=-\sin z, \quad (\sin z)'=\cos z.$$ The trigonometric addition identities
\begin{align*}
	\cos(a+b) &=\cos a \cos b-\sin a \sin b, \\
	\sin(a+b) &=\sin a \cos b+\sin b \cos a
\end{align*}
 are direct consequences of the addition identity for the exponential function.
 
 The other trigonometric functions are of tangential importance, though we remark that, for instance, $$\tan z=-i \dfrac{e^{iz}-e^{-iz}}{e^{iz}+e^{-iz}}.$$
 
 \begin{exercise}
 	Find the values of $\sin i$ and $\cos i$.
 	
 	\begin{sol}
 		We have
 		\begin{align*}
 			\cos i &=\dfrac{e^{i^2}+e^{-i^2}}{2} \\
 			&=\dfrac{e^{-1}+e}{2} \\
 			&=\dfrac{e^2+1}{2e}
 		\end{align*}
 		and
 		\begin{align*}
 			\sin i &=\dfrac{e^{i^2}-e^{-i^2}}{2i} \\
 			&=\dfrac{e^{-1}-e}{2i} \\
 			&=\dfrac{1-e^2}{2ie} \\
 			&=\left(\dfrac{e^2-1}{2e}\right)i.
 		\end{align*}
 	\end{sol}
 \end{exercise}
 
 \section{The Periodicity}
 We say that $f(z)$ has the \emph{period} $c$ if $f(z+c)=f(z)$ for all $z$. Thus a period of $e^z$ satisfies $e^{z+c}=e^z$, or $e^c=1$. It follows that $c=i\omega$ with real $\omega$; we prefer to say that $\omega$ is a period of $e^{iz}$. We shall show that there are periods, and that they are all integral multiples of a positive period $\omega_0$.
 
 Of the many ways to prove the existence of a period we choose the following: from $(\sin y)'=\cos y \le 1$ and $\sin 0=0$ we obtain $\sin y<y$ for $y>0$, as by the Mean Value Theorem, there exists $a>0$ such that $\sin y=\cos a \cdot y$ for any fixed $y>0$. Next, from $(\cos y)'=-\sin y>-y$ and $\cos 0=1$ we get that $\cos y>1-\frac{y^2}{2}$ by integrating both sides. This in turn leads to $\sin y>y-\frac{y^3}{6}$ and finally to $\cos y<1-\frac{y^2}{2}+\frac{y^4}{24}$ (note that we could also have arrived at this via the Taylor series expansion). This inequality shows that $\cos(\sqrt{3})<0$, and therefore by the Intermediate Value Theorem there is a $y_0 \in [0,\sqrt{3}]$ such that $\cos y_0=0$. Because $$\cos^2 y_0+\sin^2 y_0=1,$$ we have $\sin y_0=\pm 1$; that is $e^{iy_0}=\pm i$, and hence $e^{4iy_0}=1$. We have thus shown that $4y_0$ is a period.
 
 Actually, it is the smallest positive period. To see this, take $0<y<y_0$. Then $\sin y>y\left(1-\frac{y^2}{6}\right)>\frac{y}{2}>0$, which shows that $\cos y$ is strictly decreasing. Because $\sin y$ is positive and $\cos^2 y+\sin^2 y=1$, it follows that $\sin y$ is strictly increasing, and hence $\sin y<\sin y_0=1$. The double inequality $0<\sin y<1$ guarantees that $e^{iy}$ is neither $\pm 1$ nor $\pm i$. Therefore, $e^{4iy} \neq 1$, and $4y_0$ is indeed the smallest positive period. We denote it by $\omega_0$.
 
 Consider now an arbitrary period $\omega$. There exists an integer $n$ such that $n\omega_0 \le \omega<(n+1)\omega_0$. If $\omega$ were not equal to $n\omega_0$, then $\omega-n\omega_0$ would be another positive period smaller than $\omega_0$. Therefore, $\omega$ must be an integral multiple of $\omega_0$.
 
 \begin{definition}
 	The smallest positive period of $e^{iz}$ is denoted by $2\pi$.
 \end{definition}
 
 In the course of the proof we have shown that $e^{\pi i/2}=i$, $e^{\pi i}=-1$, and $e^{2\pi i}=1$. These are famous equations that demonstrate the intimate relationship between the numbers $e$ and $\pi$. In fact, the second equation is commonly written as $$\boxed{e^{\pi i}+1=0}$$ to include all of the ``big" constants $e,\pi,i,0,1$ in one formula. It is widely considered to be the most magical equation in all of mathematics.
 
 When $y$ increases from $0$ to $2\pi$, the point $w-e^{iy}$ describes the unit circle $\abs{\omega}=1$ in the positive sense, namely from $1$ over $i$ to $-1$ and back over $-i$ to $1$ (a counterclockwise rotation). For every $w$ with $\abs{w}=1$ there is one and only one $y$ from the half-open interval $0 \le y<2\pi$ such that $w=e^{iy}$ (such a parametrization is a diffeomorphism and can be used to construct the covering map of the unit circle).
 
 From an algebraic point of view the mapping $y \mapsto e^{iy}$ is a homomorphism from the additive group of real numbers to the multiplicative group of complex numbers with absolute value $1$. The kernel of the homomorphism is the subgroup formed by all integral multiples of $2\pi$, so we get by the First Isomorphism Theorem that $$\RR \quotient 2\pi \ZZ \cong \{w \in \CC \colon \abs{w}=1\}.$$
 
 \section{The Logarithm}
\begin{definition}
	Given $w \in \CC$, a complex number $z$ such that $e^z=w$ is called the (natural) \emph{logarithm} of $w$.
\end{definition}
First of all, the number $0$ has no logarithm. For $w \neq 0$, the equation $e^{x+iy}=w$ is equivalent to $$e^x=\abs{w}, \quad e^{iy}=w/\abs{w}.$$ The first equation has a unique solution $x=\log \abs{w}$, the \emph{real logarithm} of the positive number $\abs{w}$. The right-hand member of the second equation is a complex number of absolute value $1$. As we have seen above, it has only one solution $y$ in the half-open interval $[0,2\pi)$, but it is also satisfied by all $y+2\pi n$ for $n \in \ZZ$. We see then that \textit{every complex number other than $0$ has infinitely many logarithms which differ from each other by multiples of $2\pi i$}.

The imaginary part of $\log w$ is called the \emph{argument} of $w$, $\arg w$, and is interpreted geometrically as the \emph{angle}, measured in radians, between the positive real axis and the half-line from $)$ through the point $w$. According to this definition the argument has infinitely may values which differ by $2\pi$, and $$\log w=\log \abs{w}+i \arg w.$$ With a change of notation, if $\abs{z}=r$ and $\arg z=\theta$, then $z=re^{i \theta}$. The reader probably recalls this from high-school complex geometry.

By convention the logarithm of a positive number shall always mean the real logarithm, unless the contrary is stated. The symbol $a^b$, where $a$ and $b$ are arbitrary complex numbers except for the condition $a \neq 0$, is always interpreted as an equivalent of $\exp(b \log a)$ ($e^{b \log a}$). If $a$ is restricted to positive numbers, $\log a$ shall be real, and $a^b$ has a single value. Otherwise $\log a$ is the complex logarithm , and $a^b$ has in general infinitely many values which differ by factors $e^{2 \pi i nb}$. There will be a single value if and only if $b$ is an integer $n$, and then $a^b$ can be interpreted as a power of $a$ or $a^{-1}$. If $b$ is a rational number with the reduced form $p/q$, then $a^b$ has exactly $q$ values and can be represented as $\sqrt[q]{a^p}$.

The homomorphism represented by the exponential clearly implies that
\begin{align*}
	\log(z_1z_2) &=\log z_1+\log z_2, \\
	\arg(z_1z_2) &=\arg z_1+\arg z_2,
\end{align*}
so that $\log \colon \CC \rightarrow \CC$ is also a homomorphism from the multiplicative group of complex numbers to the additive group of complex numbers. If we want to compare a value on the left with a value on the right, then we can merely assert that they differ by a multiple of $2\pi i$ (or $2\pi$).

Finally, we discuss the inverse cosine which is obtained by solution of the equation $$\cos z=\dfrac{1}{2}\left(e^{iz}+e^{-iz}\right)=w.$$ This is a quadratic equation in $e^{iz}$ with the roots $$e^{iz}=w \pm \sqrt{w^2-1},$$ and consequently $$z= \arccos w=\cos^{-1}(w)=-i \log \left(w \pm \sqrt{w^2-1}\right).$$ We can also write these values in the form $$\arccos w=\pm i \log\left(w+\sqrt{w^2-1}\right),$$ for $w+\sqrt{w^2-1}$ and $w-\sqrt{w^2-1}$ are reciprocal numbers. The infinitely many values of $\arccos w$ now reflect the evenness and periodicity of the cosine function.

The inverse sine is more easily defined by $$\arcsin w=\dfrac{\pi}{2}-\arccos w.$$ It is worth emphasizing that in the theory of complex analytic functions, all elementary transcendental functions can thus be expressed through $e^z$ and its inverse $\log z$. In other words, there is essentially only one elementary transcendental function. Compare our development here with its more difficult counterpart through purely real analysis.

\begin{exercise}
	Prove that $3<\pi<2\sqrt{3}$.
	
	\begin{sol}
		The text has already proven the right inequality, since we showed that $\cos(\sqrt{3})<0$, so that $$2\pi<4\sqrt{3} \then \pi<2\sqrt{3}.$$ For the left inequality, we may use (after integrating again) $$\cos y>1-\dfrac{y^2}{2}+\dfrac{y^4}{24}-\dfrac{y^6}{720}.$$ Substituting $3/2$ on both sides gives
		\begin{align*}
		\cos(3/2) &<1-\dfrac{3^2}{2^2 \cdot 2}+\dfrac{3^4}{2^4 \cdot 4!}-\dfrac{3^6}{2^6 \cdot 6!} \\
		&=\dfrac{2^6 \cdot 6!-3^2 \cdot 2^4 \cdot 3 \cdot 4 \cdot 5 \cdot 6+3^4 \cdot 2^2 \cdot 5 \cdot 6-3^6}{2^6 \cdot 6!} \\
		&=\dfrac{2^6 \cdot 6!-3^2 \cdot 2^3 \cdot 6!+3^3 \cdot 6!/2-3^6}{2^6 \cdot 6!} \\
		&=\dfrac{6!(2^6-3^2 \cdot 2^3+3^3/2)-3^6}{2^6 \cdot 6!} \\
		&=\dfrac{6!(11/2)-3^6}{2^6 \cdot 6!} \\
		&>0.
		\end{align*}
		SInce we have already set $2\pi$ to be the smallest positive period of cosine, and $\frac{3}{2}<\sqrt{3}$, there cannot be any $y_0 \in [0,3/2]$ such that $\cos y_0=0$. Hence, $$2\pi>4 \cdot \dfrac{3}{2}=6 \then \pi>\dfrac{3}{2},$$ as desired.
		
		\begin{remark}
			Notice that, in the course of this proof, we have essentially shown the following stronger result:
			\begin{center}
			\textit{For real $y$, the remainder of the Taylor series expansion of both $\cos$ and $\sin$ has the same sign as its leading term}.
			\end{center}
		 	The proof is by using the fact that $(\sin y)'=\cos y$ and $(\cos y)'=-\sin y$, and integrating successive inequalities to bootstrap, as in the text and in this solution.
		\end{remark}
	\end{sol}
\end{exercise}