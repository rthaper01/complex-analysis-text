\chapter{Cauchy's Integral Formula}
\label{chap:cauchy-integral-formula}
Through a very simple application of Cauchy's theorem it becomes possible to represent an analytic function $f(z)$ as a line integral in which the variable $z$ centers as a parameter. This representation, known as \textit{Cauchy's Integral Formula}, has numerous important applications, not just in complex analysis, but also in physics and electrical engineering. Above all, it enables us to study the local properties of an analytic function in great detail.

As we declared at the beginning of this part on complex integration, we will prove Cauchy's Integral Formula purely analytically. That is, we will define winding number as an analytical, rather than a topological, quantity. For completeness, and to illustrate the connection between the two definitions, the Appendix, we will include a full topological proof of this quintessential theorem.

\section{Winding Number}
Central to many questions in function theory and physics (particularly string theory) is the notion of a curve winding around a point not on it. Let us make this idea precise, inspired by the following lemma:
\begin{lemma}
If the piecewise differentiable closed curve $\gamma$ does not pass through the point $a$, then the value of the integral $$\int_{\gamma} \dfrac{dz}{z-a}$$ is a multiple of $2\pi i$.
\end{lemma}
\begin{proof}
This lemma may seem trivial, for we can write $$\int_{\gamma}\dfrac{dz}{z-a}=\int_{\gamma} d(\log(z-a))=\int_{\gamma}d(\log \abs{z-a})+i\int_{\gamma}d(\arg(z-a)).$$ When $z$ describes a closed curve, $\log\abs{z-a}$ returns to its initial value and $\arg(z-a)$ increases or decreases by a multiple of $2\pi$. This would seem to imply the lemma, but more careful thought shows that the reasoning is of no value unless we define $\arg(z-a)$ in a unique way. We could do this, but there is an easier way to proceed.

The simplest proof is computational, exploiting the theory of first-order ordinary differential equations. If the equation of $\gamma$ is $z=z(t)$, $\alpha \le t \le \beta$, let us consider the function $$h(t)=\int_{\alpha}^{t}\dfrac{z'(t)}{z(t)-a}dt.$$ It is defined and continuous on the closed interval $[\alpha, \beta]$, and it has the derivative $$h'(t)=\dfrac{z'(t)}{z(t)-a}$$ whenever $z'(t)$ is continuous. From this equation it follows that the derivative of $e^{-h(t)}(z(t)-a)$ is $$e^{-h(t)} \cdot z'(t)-h'(t)e^{-h(t)}(z(t)-a)=e^{-h(t)} \cdot z'(t)-e^{-h(t)} \cdot z'(t)=0$$ except perhaps at a finite number of points, and since this function is continuous it must reduce to a constant.

With the initial condition $h(\alpha)=0$, we solve the differential equation to get $$e^{h(t)}=\dfrac{z(t)-a}{z(\alpha)-a}.$$ Since $z(\beta)=z(\alpha)$ we obtain $e^{h(\beta)}=1$, and therefore $h(\beta)$ must be a multiple of $2\pi i$. This proves the lemma.
\end{proof}

Motivated by this fact, we can now precisely define the winding number of a point with respect to a curve.

\begin{definition}
Given a curve $\gamma$ and a point $a$ not on it, the \emph{winding number} of $\gamma$ with respect to $a$, denoted by $W(\gamma, a)$, is defined as the integer $$W(\gamma, a)=\frac{1}{2\pi i}\int_{\gamma} \dfrac{dz}{z-a}.$$
\end{definition}

It is clear that $W(-\gamma, a)=-W(\gamma, a)$, and the following property is an immediate consequence of \ref{thm:cauchy-disk}:

\textit{if $\gamma$ lies inside of a circle, then $W(\gamma,a)=0$ for all points $a$ outside of the same circle}.

As the point set $\gamma$ is closed and bounded, its complement is open and can be represented as a union of disjoint regions, the components of the complement. We shall say, for short, that $\gamma$ determines these regions. If the complementary regions are considered in the extended plane, there is exactly one which contains the point at infinity. Consequently, $\gamma$ determines one and only one unbounded region.

\textit{As a function of $a$ the winding number $W(\gamma, a)$ is constant in each of the regions determined by $\gamma$, and zero in the unbounded region.}

To justify this, we note that any two points in the same region determined by $\gamma$ can be joined by a polygon which does not meet $\gamma$. For this reason it is sufficient to prove that $W(\gamma, a)=W(\gamma,b)$ if $\gamma$ does not meet the line segment from $a$ to $b$. Outside of this segment the function $(z-a)/(z-b)$ is never real and $\le 0$. Therefore, it is on the principal branch of $\log \left[(z-a)/(z-b)\right]$, which has derivative $(z-a)^{-1}-(z-b)^{-1}$. If $\gamma$ does not meet the segment we must have $$\int_{\gamma} \left(\dfrac{1}{z-a}-\dfrac{1}{z-b}\right)dz=0;$$ hence, $W(\gamma, a)=W(\gamma,b)$. If $\abs{a}$ is sufficiently large, $\gamma$ is contained in a disk $\abs{z}<\rho<\abs{z}$ and we conclude by the first main observation that $W(\gamma, a)=0$. Having established that the winding number is locally constant, we conclude that $W(\gamma,a)=0$ for all points $a$ in the unbounded region.

We shall find the case $W(\gamma, a)=1$ particularly important, and it is desirable to formulate a geometric condition which leads to this consequence. For simplicity we take $a=0$.

\begin{lemma}
Let $z_1,z_2$ be two points on a closed curve $\gamma$ which does not pass through the origin. Denote the subarc from $z_1$ to $z_2$ in the direction of the curve by $\gamma_1$ and the subarc from $z_2$ to $z_1$ by $\gamma_2$. Suppose that $z_1$ lies in the lower half plane and $z_2$ lies in the upper half plane. If $\gamma_1$ does not meet the negative real axis and $\gamma_2$ does not meet the positive real axis, then $W(\gamma, 0)=1$.
\end{lemma}

\begin{proof}
The proof, as is usually the case in complex analysis, is quite beautiful and elementary. Draw the half lines $L_1$ and $L_2$ from the origin through $z_1$ and $z_2$. Let $\zeta_1,\zeta_2$ be the points in which $L_1,L_2$ intersect a circle $C$ about the origin. If $C$ is described in the positive sense, the arc $C_1$ from $\zeta_1$ to $\zeta_2$ does not intersect the negative axis, and the arc $C_2$ from $\zeta_2$ to $\zeta_1$ does intersect the positive axis (since $z_1$ lies in the lower half plane while $z_2$ lies in the upper half plane). Denote the directed line segments from $z_1$ to $\zeta_1$ and from $z_2$ to $\zeta_2$ by $\delta_1$ and $\delta_2$, respectively. Introducing the closed curves $\sigma_1=\gamma_1+\delta_2-C_1-\delta_1$ and $\sigma_2=\gamma_2+\delta_1-C_2-\delta_2$, we find that 
\begin{align*}
W(\gamma, 0) &= W(C_1,0)+W(C_2,0)+W(\sigma_1,0)+W(\sigma_2,0) \\
&\quad -W(\delta_2,0)+W(\delta_1,0)+W(\delta_2,0)-W(\delta_1,0) \\
&= W(C,0)+W(\sigma_1,0)+W(\sigma_2,0).
\end{align*} But $\sigma_1$ does not meet the negative axis. Hence the origin belongs to the unbounded region determined by $\sigma_1$, and therefore $W(\sigma_1,0)=0$. Similarly, $\sigma_2$ does not meet the positive axis, so $W(\sigma_2,0)=0$. Finally, since $C$ is a circle about the origin, $W(C,0)=1$. Thus we conclude that $$W(\gamma, 0)=1+0+0=1.$$

\begin{figure}[h]
\centering
\begin{asy}
import graph;
// Custom pens (not built-in) to avoid undefined identifier errors
pen heavygreen = rgb(0,0.55,0);
pen orange = rgb(1,0.55,0);
size(10cm);

// Set up coordinate system
real xmin = -3, xmax = 3, ymin = -2.5, ymax = 2.5;

// Draw axes
draw((xmin,0)--(xmax,0), gray+0.5bp, Arrow);
draw((0,ymin)--(0,ymax), gray+0.5bp, Arrow);

// Define key points
pair z1 = (2.2, -1.5);  // Point in lower half-plane
pair z2 = (-1.8, 2.0);  // Point in upper half-plane
pair origin = (0, 0);

// Draw circle C about origin through a convenient radius
real radius = 1.0;
path circle_C = circle(origin, radius);
draw(circle_C, black+0.5bp);
label("$C$", (radius*cos(pi/6), radius*sin(pi/6)), NE);

// Draw half-lines L1 and L2 from origin through z1 and z2
real line_length = 2.5;
pair L1_end = origin + line_length * unit(z1);
pair L2_end = origin + line_length * unit(z2);

draw(origin--L1_end, dashed+gray+0.5bp);
draw(origin--L2_end, dashed+gray+0.5bp);
label("$L_1$", L1_end + (0.1, -0.1), SE);
label("$L_2$", L2_end + (-0.1, 0.1), NW);

// Find intersection points ζ₁ and ζ₂ of half-lines with circle
pair zeta1 = origin + radius * unit(z1);
pair zeta2 = origin + radius * unit(z2);

// Draw and label the intersection points
dot(zeta1, black+3bp);
dot(zeta2, black+3bp);
label("$\zeta_1$", zeta1 + (0.15, 0.15), NE);
label("$\zeta_2$", zeta2 + (-0.25, 0.15), NW);

// Draw the closed curve γ
// γ₁: from z1 to z2 (avoiding negative real axis)
path gamma1 = z1{N+E}..{W+N}(1.5, 1.2)..{W+N}(0.3, 1.8)..{W+N}z2;
draw(gamma1, blue+0.7bp, Arrow(Relative(0.7)));

// γ₂: from z2 to z1 (avoiding positive real axis)  
path gamma2 = z2{S+W}..{E+S}(-1.2, -0.8)..{E+S}(0.5, -1.5)..{E+S}z1;
draw(gamma2, red+0.7bp, Arrow(Relative(0.7)));

// Draw and label the curve points
dot(z1, black+2bp);
dot(z2, black+2bp);
label("$z_1$", z1 + (0, -0.2), S);
label("$z_2$", z2 + (0, 0.2), N);

// Label the curve segments
label("$\gamma_1$", (0.5, 2.2), blue);
label("$\gamma_2$", (-1.5, -1.8), red);
label("$\gamma$", (2.7, 0.2), black);

// Draw the line segments δ₁ and δ₂
draw(z1--zeta1, black+0.5bp+dashed);
draw(z2--zeta2, black+0.5bp+dashed);
label("$\delta_1$", (z1+zeta1)/2 + (0.2, 0), E);
label("$\delta_2$", (z2+zeta2)/2 + (-0.2, 0), W);

// Draw arcs C₁ and C₂ on the circle
// C₁: from ζ₁ to ζ₂ (not intersecting negative axis)
real angle1 = atan2(zeta1.y, zeta1.x);
real angle2 = atan2(zeta2.y, zeta2.x);

// Ensure we go the shorter way that avoids negative axis
if (angle2 < angle1) angle2 += 2*pi;
path arc_C1 = arc(origin, radius, degrees(angle1), degrees(angle2));
draw(arc_C1, heavygreen+0.7bp, Arrow(Relative(0.5)));
label("$C_1$", radius*(cos((angle1+angle2)/2), sin((angle1+angle2)/2)) + (0.2, 0.1), align=NoAlign, heavygreen);

// C₂: from ζ₂ to ζ₁ (the complementary arc, intersecting positive axis)
path arc_C2 = arc(origin, radius, degrees(angle2), degrees(angle1+2*pi));
draw(arc_C2, orange+0.7bp, Arrow(Relative(0.5)));
label("$C_2$", radius*(cos((angle1+angle2)/2 + pi), sin((angle1+angle2)/2 + pi)) + (-0.2, -0.1), align=NoAlign, orange);

// Add origin point
dot(origin, black+2bp);
label("$0$", origin + (-0.15, -0.15), SW);
\end{asy}
\end{figure}
\end{proof}

This proof is a brilliant example of when analytical proofs trump topological ones. Had we started off by defining winding numbers entirely topologically, we would have had to digress into homotopy invariance, differential forms, and Stokes' Theorem, then proven the degree formula to relate the topological and analytical definitions of winding number, and only then prove the integral formula. Along with all of this, we would have to define some more abstract notion of orientation in terms of equivalence classes of bases of tangent spaces to smooth manifolds, rather than sticking to the simple, easily visualized concepts of left and right. All of this is very doable, beautiful in its own way, and is the more general and correct course of action to perform analysis on higher-dimensional manifolds. But we do not need any of it for complex analysis. See the Appendix for a glimpse at the topological machinery.

\section{The Integral Formula}
We are now ready to state and prove Cauchy's Integral Formula.

\begin{theorem}[Cauchy's Integral Formula]
\label{thm:cauchy-integral-formula}
Suppose that $f(z)$ is analytic in an open disk $\Delta$, and let $\gamma$ be a closed curve in $\Delta$. For any point $a$ not on $\gamma$, we have $$W(\gamma, a) \cdot f(a)=\dfrac{1}{2\pi i}\int_{\gamma} \dfrac{f(z)dz}{z-a}.$$
\end{theorem}
\begin{proof}
If $a \notin \Delta$, then $W(\gamma, a)=0$ and $\int_{\gamma} \frac{f(z)dz}{z-a}=0$ since the integrand is analytic in $\Delta$, so the theorem is vacuously true regardless of what value $f$ takes on at $a$. We may therefore assume that $a \in \Delta$.

We apply Cauchy's Theorem to the function $$F(z)=\dfrac{f(z)-f(a)}{z-a}$$, which is analytic for all $z \neq a$. At $z=a$ it is not defined, but it satisfies the condition $$\lim_{z \rightarrow a}F(z)(z-a)=\lim_{z \rightarrow a}(f(z)-f(a))=0,$$ which is the condition of Theorem \ref{thm:cauchy-disk-stronger}. We conclude that $$\int_{\gamma} \dfrac{f(z)-f(a)}{z-a}dz=0.$$ This can be written as $$\int_{\gamma} \dfrac{f(z)dz}{z-a}=f(a)\int_{\gamma} \dfrac{dz}{z-a},$$ and the integral on the right-hand side is $2\pi i \cdot W(\gamma, a)$. This completes the proof.
\end{proof}

The most common application of Cauchy's Integral Formula is the case $W(\gamma, a)=1$; we have then $$f(a)=\dfrac{1}{2\pi i}\int_{\gamma} \dfrac{f(z)dz}{z-a},$$ and this we interpret as a \emph{representation formula}. Indeed, it permits us to compute $f(a)$ as soon as the values of $f(z)$ on $\gamma$ are given, together with the fact that $f(z)$ is analytic in $\Delta$. Letting $a$ take different values, provided that its winding number with respect to $\gamma$ remains $1$, we can treat it as a parameter in and of itself, and change the notation to write
\begin{equation}
\label{eq:representation-formula}
f(z)=\dfrac{1}{2\pi i}\int_{\gamma} \dfrac{f(\zeta)d\zeta}{\zeta-z}.
\end{equation} It is this formula that is usually referred to as \emph{Cauchy's Integral Formula}. We must remember that it is valid only for points $z$ in the region determined by $\gamma$ for which $W(\gamma, z)=1$, and that we have proved it only when $f(z)$ is analytic in a disk.

Cauchy's Integral Formula forms part of a class of results that all end up, in some way or another, extrapolating information from the boundary of a domain to its interior. At the risk of sounding like a broken record, we allude yet again to topology and the classic theorems of Green and Stokes as well as the Boundary Theorem from intersection theory.

\begin{exercise}
Compute the following integrals.
\begin{enumerate}
\item[(a)] $\int_{\abs{z}=1}\dfrac{e^z}{z}dz$.
\item[(b)] $\int_{\abs{z}=2} \dfrac{dz}{z^2+1}$.
\item[(c)] $\int_{\abs{z}=\rho}\dfrac{\abs{dz}}{\abs{z-a}^2}$.
\end{enumerate}

\begin{sol}
$ $
\begin{enumerate}
\item[(a)] The answer is $\boxed{2\pi i}$. Since $e^z$ is analytic everywhere, by Cauchy's Integral Formula, we have $$\int_{\abs{z}=1}\dfrac{e^z}{z}dz=2\pi i \cdot W(\{\abs{z}=1\},0) \cdot e^0=1 \cdot 1=2\pi i.$$
\item[(b)] The answer is $\boxed{0}$. Decompose the integrand into partial fractions: $$\int_{\abs{z}=2} \dfrac{dz}{z^2+1}=\dfrac{1}{2i}\left(\int_{\abs{z}=2}\dfrac{dz}{z-i}-\int_{\abs{z}=2}\dfrac{dz}{z+i}\right).$$ Since $f(z)=1$ is analytic everywhere,
\begin{align*}
\int_{\abs{z}=2}\dfrac{dz}{z-i} &= 2\pi i \cdot W(\{\abs{z}=2\},i)=2\pi i,\\
\int_{\abs{z}=2}\dfrac{dz}{z+i} &= 2\pi i \cdot W(\{\abs{z}=2\},-i)=2\pi i.
\end{align*}
The winding number of any point inside the circle is $1$, so the two terms cancel and the integral is $0$.
\item[(c)] The answer is $\boxed{\dfrac{2\pi \rho}{\abs{\rho^2-\abs{a}^2}}}$.

This is a fairly involved computational problem. The key insight is to express the real-valued arc-length integral as a complex integral so that we can apply Cauchy's Integral Formula.

First, observe that on the circle $\abs{z}=\rho$, parameterized as $z(t)=\rho e^{it}$, $0 \le t \le 2\pi$, we have
\begin{align*}
\abs{dz} &=\abs{z'(t)}dt \\
&=\rho dt \\
&=-i\rho \cdot i dt \\
&=-i \rho \cdot \dfrac{i \rho e^{it}}{\rho e^{it}}dt \\
&=-i \rho \cdot \dfrac{dz}{z}.
\end{align*}
Hence, we can reframe the integral as
\begin{align*}
\int_{\abs{z}=\rho}\dfrac{\abs{dz}}{\abs{z-a}^2} &= -i \rho\int_{\abs{z}=\rho}\dfrac{dz}{z\abs{z-a}^2} \\
&=-i\rho \int_{\abs{z}=\rho}\dfrac{dz}{z(z-a)(\overline{z}-\overline{a})} \\
&=-i\rho \int_{\abs{z}=\rho}\dfrac{dz}{z(z\overline{z}-\overline{a}z-a\overline{z}+a\overline{a})} \\
&=-i\rho \int_{\abs{z}=\rho}\dfrac{dz}{z(\rho^2-\overline{a}z-a\overline{z}+\abs{a}^2)} \\
&=-i\rho \int_{\abs{z}=\rho}\dfrac{dz}{-\overline{a}z^2+(\rho^2+\abs{a}^2)z-a\rho^2} \\
&=i\rho \int_{\abs{z}=\rho}\dfrac{dz}{\overline{a}z^2+(\rho^2-\abs{a}^2)z+a\rho^2}.
\end{align*}
Let us now decompose the denominator into partial fractions. The roots of the quadratic polynomial $\overline{a}z^2-(\rho^2+\abs{a}^2)z+a\rho^2$ are
\begin{align*}
z &=\dfrac{\rho^2+\abs{a}^2 \pm \sqrt{(\rho^2+\abs{a}^2)^2-4\overline{a}a\rho^2}}{2\overline{a}} \\
&=\dfrac{\rho^2+\abs{a}^2 \pm \sqrt{(\rho^2+\abs{a}^2)^2-4\abs{a}^2\rho^2}}{2\overline{a}} \\
&=\dfrac{\rho^2+\abs{a}^2 \pm (\rho^2-\abs{a}^2)}{2\overline{a}} \\
&=a, \dfrac{\rho^2}{\overline{a}}.
\end{align*}
These are symmetric with respect to the circle $\abs{z}=\rho$. The partial fraction decomposition of $\frac{1}{\overline{a}\left(z-a\right)\left(z-\rho^2/\overline{a}\right)}$ is $$\dfrac{1}{\rho^2-\abs{a}^2}\left(\dfrac{1}{z-\frac{\rho^2}{\overline{a}}}-\dfrac{1}{z-a}\right).$$ Consider the cases (1) $\abs{a}<\rho$ and (2) $\abs{a}>\rho$. In the first case, $\rho^2/\overline{a}$ is outside the circle, $W(\{\abs{z}=\rho\},\rho^2/\overline{a})=0$, $W(\{\abs{z}=\rho\},a)=1$, and the integral equals
\begin{align*}
\dfrac{i\rho}{\rho^2-\abs{a}^2}\left(\int_{\abs{z}=\rho}\dfrac{dz}{z-\frac{\rho^2}{\overline{a}}}-\int_{\abs{z}=\rho}\dfrac{dz}{z-a}\right) &= \dfrac{i\rho}{\rho^2-\abs{a}^2}\left(0-2\pi i\right) \\
&=\frac{2\pi \rho}{\rho^2-\abs{a}^2}.
\end{align*}
In case (2), $a$ is outside the circle, $W(\{\abs{z}=\rho\},\rho^2/\overline{a})=1$, $W(\{\abs{z}=\rho\},a)=0$, and the integral equals
\begin{align*}
\dfrac{i\rho}{\rho^2-\abs{a}^2}\left(\int_{\abs{z}=\rho}\dfrac{dz}{z-\frac{\rho^2}{\overline{a}}}-\int_{\abs{z}=\rho}\dfrac{dz}{z-a}\right) &= \dfrac{i\rho}{\rho^2-\abs{a}^2}\left(2\pi i-0\right) \\
&=-\frac{2\pi \rho}{\rho^2-\abs{a}^2}.
\end{align*}
In either case we take the positive value of $\rho^2-\abs{a}^2$, giving the unified answer $$\dfrac{2\pi \rho}{\abs{\rho^2-\abs{a}^2}}.$$
\end{enumerate}
\end{sol}
\end{exercise}

\section{Higher Derivatives}
The representation formula \ref{eq:representation-formula} gives us an ideal tool for the study of the local properties of analytic functions. In particular, we can now show that an analytic function has derivatives of all orders, which are then also analytic.

We consider a function $f(z)$ which is analytic in an arbitrary region $\Omega$. To a point $a \in \Omega$ we determine an $\eps$-neighborhood $\Delta$ contained in $\Omega$, and in $\Delta$ a circle $C$ about $a$ (this is just using the fact that an open set contains an open ball around each point). Theorem \ref{thm:cauchy-integral-formula} can be applied to $f(z)$ in $\Delta$. Since $W(C,a)=1$ and the winding number is locally constant, we have $W(C,z)=1$ for all points $z$ inside of $C$. For such $z$ we obtain by \ref{eq:representation-formula} $$f(z)=\dfrac{1}{2\pi i}\int_{C} \dfrac{f(\zeta)d\zeta}{\zeta-z}.$$ Provided that the integral can be differentiated under the sign of integration we find
\begin{equation}
\label{eq:first-derivative}
f'(z)=\dfrac{1}{2\pi i}\int_{C} \dfrac{f(\zeta)d\zeta}{(\zeta-z)^2}
\end{equation}
and 
\begin{equation}
\label{eq:higher-derivatives}
f^{(n)}(z)=\dfrac{n!}{2\pi i}\int_{C} \dfrac{f(\zeta)d\zeta}{(\zeta-z)^{n+1}}.
\end{equation}
If the differentations can be justified, we shall have proved the existence of all derivatives at the points inside of $C$. Since every point in $\Omega$ lies inside of some such circle, the existence will be proved in the whole region $\Omega$. At the same time, we shall have obtained a convenient representation formula for the derivatives.

For the justification we could either refer to corresponding theorems in the real case, or we could prove a general theorem concerning line integrals whose integrand depends analytically on a parameter. Actually, we shall prove only the following lemma which is all we need in the present case:
\begin{lemma}
\label{lem:derivative-integral}
Suppose that $\varphi(\zeta)$ is continuous on the arc $\gamma$. Then the function $$F_n(z)=\int_{\gamma}\dfrac{\varphi(\zeta)d\zeta}{(\zeta-z)^n}$$ is analytic in each of the regions determined by $\gamma$, and its derivative is $F_n'(z)=nF_{n+1}(z)$.
\end{lemma}

\begin{proof}
The proof is by induction on $n$.

We prove first that $F_1(z)$ is continuous. Let $z_0$ be a point not on $\gamma$, and choose the neighborhood $\abs{z-z_0}<\delta$ so that it does not meet $\gamma$ (we know that we can do this because the complement of $\gamma$ is an open set). Restrict $z$ to the smaller neighborhood $\abs{z-z_0}<\delta/2$. Then, we get that \begin{align*}
\abs{\zeta-z} &\ge \abs{\zeta-z_0}-\abs{z-z_0} \\
&>\delta-\dfrac{\delta}{2} \\
&=\dfrac{\delta}{2}
\end{align*}
for all $\zeta \in \gamma$. From $$F_1(z)-F_1(z_0)=(z-z_0)\int_{\gamma}\dfrac{\varphi(\zeta)d\zeta}{(\zeta-z)(\zeta-z_0)}$$ we obtain at once by another application of the Triangle Inequality for integrals
\begin{equation}
\label{eq:integral-continuous}
\abs{F_1(z)-F_1(z_0)}\le \abs{z-z_0} \cdot \dfrac{2}{\delta^2}\int_{\gamma}\abs{\varphi} \abs{d\zeta},
\end{equation} so $F_1$ is Lipschitz-continuous.

The bound in \ref{eq:integral-continuous} allows us to apply the dominated convergence theorem from measure theory to the difference quotient $$\dfrac{F_1(z)-F_1(z_0)}{z-z_0}=\int_{\gamma}\dfrac{\varphi(\zeta)d\zeta}{(\zeta-z)(\zeta-z_0)}.$$ As $z \rightarrow z_0$, we may pass the limit into the integral to get that the left-hand side tends to $F_2(z_0)$. This proves the base case.

Assume now for the inductive hypothesis that $F_{n-1}'(z)=(n-1)F_n(z)$. First, consider the general algebraic identity 
\begin{align*}
\dfrac{y^n-x^n}{x^ny^n} &=\dfrac{xy^{n-1}-x^n+y^n-xy^{n-1}}{x^ny^n} \\
&=\dfrac{xy^{n-1}-x^n+(y-x)y^{n-1}}{x^ny^n} \\
&=\dfrac{1}{x^{n-1}y}-\dfrac{1}{y^n}+\dfrac{y-x}{x^ny}.
\end{align*}
Inserting $\zeta-z_0$ for $y$ and $\zeta-z$ for $x$ multiplying by $\phi$, and taking the integral over $d\zeta$ gives\begin{align*}
F_n(z)-F_n(z_0) &=\left[\int_{\gamma} \dfrac{\varphi(\zeta)d\zeta}{(\zeta-z)^{n-1}(\zeta-z_0)} - \int_{\gamma} \dfrac{\varphi(\zeta)d\zeta}{(\zeta-z_0)^n}\right]+(z-z_0)\int_{\gamma}\dfrac{\varphi d\zeta}{(\zeta-z)^n(\zeta-z_0)}.
\end{align*}
By the inductive hypothesis, the first addend tends to $0$ as $z \rightarrow z_0$, and in the second term the factor of $z-z_0$ is bounded in a neighborhood of $z_0$. Hence, $F_n(z)$ is continuous.

Now, if we divide the identity by $z-z_0$ and let $z$ tend to $z_0$, the quotient in the first term tends to a derivative which by the induction hypothesis equals $(n-1)F_{n+1}(z_0)$. The remaining factor in the second term is continuous, by what we have already proved, and has the limit $F_{n+1}(z_0)$. Hence, $F_n'(z_0)$ exists and equals $nF_{n+1}(z_0)$.
\end{proof}

It is clear that Lemma \ref{lem:derivative-integral} is just what is needed in order to deduce \ref{eq:first-derivative} and \ref{eq:higher-derivatives} in a rigorous way. We have thus proved that an analytic function has derivatvies of all orders which are analytic and can be represented by the formula \ref{eq:higher-derivatives}.

Among the consequences of this result we'd like to single out two classical theorems.

\begin{theorem}[Morera]
If $f(z)$ is defined and continuous in a region $\Omega$, and if $\int_{\gamma} f dz=0$ for all closed curves $\gamma$ in $\Omega$, then $f(z)$ is analytic in $\Omega$.
\end{theorem}
\begin{proof}
The hypothesis implies that $f(z)$ is the derivative of an analytic function $F(z)$. We now know that $f$ itself is analytic in $\Omega$.
\end{proof}

\begin{theorem}[Liouville]
A function which is analytic and bounded in the whole plane must reduce to a constant.
\end{theorem}
\begin{proof}
We make use of a simple estimate derived from \ref{eq:higher-derivatives}, known as \emph{Cauchy's estimate}. Let the radius of $C$ be $r$, and assume that $\abs{f(\zeta)} \le M$ on $C$. If we apply \ref{eq:higher-derivatives} with $z=a$, we obtain at once
\begin{align*}
\abs{f^{(n)}(a)} &= \left\abs{\dfrac{n!}{2\pi i}\int_{C} \dfrac{f(\zeta)d\zeta}{(\zeta-a)^{n+1}}\right} \\
&\le \dfrac{M n!}{2\pi} \cdot \int_{C}\dfrac{\abs{d\zeta}}{\abs{\zeta-a}^{n+1}} \\
&=\dfrac{M n!}{2\pi} \cdot \int_{C}\abs{d\zeta}r^{-(n+1)} \\
&=\dfrac{M n!}{2\pi} \cdot \int_{0}^{2\pi}r^{-n}dt \\
&=Mn!r^{-n}.
\end{align*}
For Liouville's theorem we only need the case $n=1$. The hypothesis means that $\abs{f(\zeta)} \le M$ on all circles. Hence we can let $r$ tend to $\infty$, and the estimate leads to $f'(a)=0$ for all $a$. We conclude that the function is constant.hvfffcvc
\end{proof}
Liouville's theorem leads to an almost trivial proof of the Fundamental Theorem of Algebra. Suppose that $P(z)$ is a polynomial of degree $>0$. If $P(z)$ were never zero, the function $1/P(z)$ would be analytic in the whole plane. We know that $P(z) \rightarrow \infty$ for $z \rightarrow \infty$, and therefore $1/P(z)$ tends to zero. This implies boundedness (the absolute value is continuous on the compact Riemann sphere and thus has a finite maximum), and by Liouville's theorem $1/P(z)$ would be constant. But this is absurd, so the equation $F(z)=0$ must have a root.

The reader is free to choose between this purely analytical proof or the topological one by Milnor (1963). Both are quite beautiful.

Cauchy's estimate shows that the successive derivatives of an analytic function cannot be arbitrary; there must always exist an $M$ and an $r$ so that the inequality is fulfilled. In order to make the best use of the inequality it is important that $r$ be judiciously chosen, the object being to minimize the function $M(r)r^{-n}$, where $M(r)$ is the maximum of $\abs{f}$ on $\abs{\zeta-a}=r$.

\begin{exercise}
Compute
\begin{enumerate}
\item[(a)] $\int_{\abs{z}=1}e^z z^{-n}dz$,
\item[(b)] $\int_{\abs{z}=\rho} \abs{z-a}^{-4}\abs{dz}$ for $\abs{a} \neq \rho$.
\end{enumerate}
\begin{sol}
$ $
\begin{enumerate}
\item[(a)] The answer is $\dfrac{2\pi i}{(n-1)!}$.

The function $f(z)=e^z$ is obviously analytic in any disk about $0$, so we have $$\int_{\abs{z}=1}e^z z^{-n}=\dfrac{2\pi i f(0)}{(n-1)!}=\dfrac{2\pi i}{(n-1)!}.$$

\item[(b)] The answer is $\boxed{\dfrac{2\pi \rho (\rho^2+\abs{a}^2)}{\abs{\rho^2-\abs{a}^2}^3}}$.

Recalling the exercise at the end of the last section, we transform the integrand:
\begin{align*}
\dfrac{\abs{dz}}{\abs{z-a}^4} &=-i\rho \cdot \dfrac{dz}{(z-a)^2(\overline{a}z-\rho^2)(\overline{z}-\overline{a})}.
\end{align*}
The second and third factors in the denominator can be refactored as follows:
\begin{align*}
\left(\overline{a}z-\rho^2\right)\left(\overline{z}-\overline{a}\right) &=\overline{a}\rho^2-\overline{a}^2z-\rho^2\overline{z}+\overline{a}\rho^2 \\
&=2\overline{a}\rho^2-\overline{a}^2z-\dfrac{\rho^4}{z} \\
&=\dfrac{2\overline{a}\rho^2z-\overline{a}^2z^2-\rho^4}{z} \\
&=-\dfrac{(\rho^2-\overline{a}z)^2}{z}.
\end{align*}
Hence, $$\dfrac{\abs{dz}}{\abs{z-a}^4}=\dfrac{i\rho}{\overline{a}^2} \cdot \dfrac{zdz}{(z-a)^2(z-\rho^2/\overline{a})^2}.$$ For convenience, set $a^*=\rho^2/a$ (it is the point of symmetry of $a$ with respect to the circle $\abs{z}=\rho$). We decompose the integrant into partial fractions as follows:
\begin{align*}
\dfrac{z}{(z-a)^2(z-a^*)^2} &=\dfrac{A_1}{z-a}+\dfrac{A_2}{(z-a)^2}+\dfrac{B_1}{z-a^*}+\dfrac{B_2}{(z-a^*)^2} \\
\then z &=A_1(z-a)(z-a^*)^2+A_2(z-a^*)^2 \\ &+B_1(z-a)^2(z-a^*)+B_2(z-a)^2.
\end{align*}
Plugging in $a$ and $a^*$ immediately gives $$A_2=\dfrac{a}{(a-a^*)^2}, \quad B_2=\dfrac{a^*}{(a-a^*)^2}.$$ Next, taking the derivative of both sides yields
\begin{align*}
1 &=A_1\left(2(z-a)(z-a^*)+(z-a^*)^2\right)+2A_2(z-a^*) \\ &+B_1(2(z-a)(z-a^*)+(z-a)^2)+2B_2(z-a)^2.
\end{align*}
Plugging in $a$ and $a^*$ into the above equation, we get $$A_1=-\dfrac{a+a^*}{(a-a^*)^3}, \quad B_1=\dfrac{a+a^*}{(a-a^*)^3}.$$ Finally, we can write the integral, decomposed into partial fractions, as
\begin{align*}
\dfrac{\abs{dz}}{\abs{z-a}^4} &=\dfrac{i\rho}{\overline{a}^2(a-a^*)^3}\left(\int_{\abs{z}=\rho}\dfrac{a(a-a^*)dz}{(z-a)^2}+\int_{\abs{z}=\rho}\dfrac{a^*(a-a^*)dz}{(z-a^*)^2}\right) \\ &+\dfrac{i\rho(a+a^*)}{\overline{a}^2(a-a^*)^3}\left(\int_{\abs{z}=\rho}\dfrac{dz}{z-a^*}-\int_{\abs{z}=\rho}\dfrac{dz}{z-a}\right) \\
&=\dfrac{i\rho(a+a^*)}{\overline{a}^2(a-a^*)^3}\left(\int_{\abs{z}=\rho}\dfrac{dz}{z-a^*}-\int_{\abs{z}=\rho}\dfrac{dz}{z-a}\right) \\
&=-\dfrac{i\rho(\rho^2+\abs{a}^2)}{(\rho^2-\abs{a}^2)^3}\left(\int_{\abs{z}=\rho}\dfrac{dz}{z-\rho^2/\overline{a}}-\int_{\abs{z}=\rho}\dfrac{dz}{z-a}\right) \\
&=\dfrac{2\pi \rho (\rho^2+\abs{a}^2)}{\abs{\rho^2-\abs{a}^2}^3},
\end{align*}
where in the third step we used Cauchy's formula for higher derivatives to eliminate the integrals in the first term.
\end{enumerate}
\end{sol}
\end{exercise}
\begin{exercise}
Show that the successive derivatives of an analytic function at a point can never satisfy $\abs{f^{(n)}(z)}>n! n^n$ for all $n$.

\begin{sol}
Assume for the sake of contradiction that $\abs{f^{(n)}(z)}>n! n^n$ for all $n$. Since $f$ is analytic at $0$, it is analytic in some closed disk of radius $\eps$ about $0$. On the circle $\abs{z}=\eps$, which is compact, $\abs{f}$ is bounded by some $M$, so by Cauchy's estimate, we have $$Mn!\eps^{-n}>n! n^n \then M>(\eps n)^n$$ for all $n$. But this is impossible since $M$ is finite, so we conclude that $\abs{f^{(n)}(z)}$ cannot be greater than $n! n^n$ for all $n$.
\end{sol}
\end{exercise}
\begin{exercise}
Prove that a function which is analytic in the whole plane and satisfies an equality $\abs{f(z)}<\abs{z}^n$ for some $n$ and all sufficiently large $\abs{z}$ reduces to a polynomial.

\begin{sol}
Let $R$ be such that for all $\abs{z} \ge R$, the inequality holds; that is, $\abs{f(z)}<R^n$ on the circle $\abs{z}=R$. Using Cauchy's estimate, we have that $$\abs{f^{(m)}(0)}<m! R^{n-m}.$$ Letting $R \rightarrow \infty$, we see that for $m>n$, $f'(0)=0$. Hence, the Taylor series expansion of $f$ about $0$ terminates, and so $f$ can be written as $$f(z)=\sum_{m=0}^{n}\dfrac{f^{(m)}(0)}{m!}z^m,$$ which is a polynomial of degree at most $n$.
\end{sol}
\end{exercise}

