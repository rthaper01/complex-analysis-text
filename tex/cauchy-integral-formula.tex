\chapter{Cauchy's Integral Formula}
\label{chap:cauchy-integral-formula}
Through a very simple application of Cauchy's theorem it becomes possible to represent an analytic function $f(z)$ as a line integral in which the variable $z$ centers as a parameter. This representation, known as \textit{Cauchy's Integral Formula}, has numerous important applications, not just in complex analysis, but also in physics and electrical engineering. Above all, it enables us to study the local properties of an analytic function in great detail.

As we declared at the beginning of this part on complex integration, we will prove Cauchy's Integral Formula purely analytically. That is, we will define winding number as an analytical, rather than a topological, quantity. For completeness, and to illustrate the connection between the two definitions, the Appendix, we will include a full topological proof of this quintessential theorem.

\section{Winding Number}
Central to many questions in function theory and physics (particularly string theory) is the notion of a curve winding around a point not on it. let us make this idea precise, inspired by the following lemma:
\begin{lemma}
If the piecewise differentiable closed curve $\gamma$ does not pass through the point $a$, then the value of the integral $$\int_{\gamma} \dfrac{dz}{z-a}$$ is a multiple of $2\pi i$.
\end{lemma}
\begin{proof}
This lemma may seem trivial, for we can write $$\int_{\gamma}\dfrac{dz}{z-a}=\int_{\gamma} d(\log(z-a))=\int_{\gamma}d(\log \abs{z-a})+i\int_{\gamma}d(\arg(z-a)).$$ When $z$ describes a closed durve, $\log\abs{z-a}$ returns to its intial value and $\arg(z-a)$ increases or decreases by a multiple of $2\pi$. This would seem to imply the lemma, but more careful thought shows that the reasoning is of no value unless we define $\arg(z-a)$ in a unique way. We could do this, but there is an easier way to proceed.

The simplest proof is computational, exploiting the theory of first-order ordinary differential equations. If the equation of $\gamma$ is $z=z(t)$, $\alpha \le t \le \beta$, let us consider the function $$h(t)=\int_{\alpha}^{t}\dfrac{z'(t)}{z(t)-a}dt.$$ It is defined and continuous on the closed interval $[\alpha, \beta]$, and it has the derivative $$h'(t)=\dfrac{z'(t)}{z(t)-a}$$ whenever $z'(t)$ is continuous. From this equation it follows that the derivative of $e^{-h(t)}(z(t)-a)$ is $$e^{-h(t)} \cdot z'(t)-h'(t)e^{-h(t)}(z(t)-a)=e^{-h(t)} \cdot z'(t)-e^{-h(t)} \cdot z'(t)=0$$ except perhaps at a finite number of points, and since this function is continuous it must reduce to a constant.

With the initial condition $h(\alpha)=0$, we solve the differential equation to get $$e^{h(t)}=\dfrac{z(t)-a}{z(\alpha)-a}.$$ Since $z(\beta)=z(\alpha)$ we obtain $e^{h(\beta)}=1$, and therefore $h(\beta)$ must be a multiple of $2\pi i$. This proves the lemma.
\end{proof}

Motivated by this fact, we can now precisely define the winding number of a point with respect to a curve.

\begin{definition}
Given a curve $\gamma$ and a point $a$ not on it, the \emph{winding number} of $\gamma$ with respect to $a$, denoted by $W(\gamma, a)$, is defined as the integer $$W(\gamma, a)=\frac{1}{2\pi i}\int_{\gamma} \dfrac{dz}{z-a}.$$
\end{definition}

It is clear that $W(-\gamma, a)=-W(\gamma, a)$, and the following property is an immediate consequence of \ref{thm:cauchy-disk}:

\textit{if $\gamma$ lies inside of a circle, then $W(\gamma,a)=0$ for all points $a$ outside of the same circle}.

As the point set $\gamma$ is closed and bounded, its complement is open and can be represented as a union of disjoint regions, the components of the complement. We shall say, for short, that $\gamma$ determines these regions. If the complementary regions are considered in the extended plane, there is exactly one which contains the point at infinity. Consequently, $\gamma$ determines one and only one unbounded region.

\textit{As a function of $a$ the winding number $W(\gamma, a)$ is constant in each of the regions determined by $\gamma$, and zero in the unbounded region.}

To justify this, we note that any two points in the same region determined by $\gamma$ can be joined by a polygon which does not meet $\gamma$. For this reason it is sufficient to prove that $W(\gamma, a)=W(\gamma,b)$ if $\gamma$ does not meet the line segment from $a$ to $b$. Outside of this segment the function $(z-a)/(z-b)$ is never real and $\le 0$. Therefore, it is on the principal branch of $\log \left[(z-a)/(z-b)\right]$, which has derivative $(z-a)^{-1}-(z-b)^{-1}$. If $\gamma$ does not meet the segment we must have $$\int_{\gamma} \left(\dfrac{1}{z-a}-\dfrac{1}{z-b}\right)dz=0;$$ hence, $W(\gamma, a)=W(\gamma,b)$. If $\abs{a}$ is sufficiently large, $\gamma$ is contained in a disk $\abs{z}<\rho<\abs{z}$ and we conclude by the first main observation that $W(\gamma, a)=0$. Having established that the winding number is locally constant, we conclude that $W(\gamma,a)=0$ for all points $a$ in the unbounded region.

We shall find the case $W(\gamma, a)=1$ particularly important, and it is desirable to formulate a geometric condition which leads to this consequence. For simplicity we take $a=0$.

\begin{lemma}
Let $z_1,z_2$ be two points on a closed curve $\gamma$ which does not pass through the origin. Denote the subarc from $z_1$ to $z_2$ in the direction of the curve by $\gamma_1$ and the subarc from $z_2$ to $z_1$ by $\gamma_2$. Suppose that $z_1$ lies in the lower half plane and $z_2$ lies in the upper half plane. If $\gamma_1$ does not meet the negative real axis and $\gamma_2$ does not meet the positive real axis, then $W(\gamma, 0)=1$.
\end{lemma}

\begin{proof}
The proof, as is usually the case in complex analysis, is quite beautiful and elementary. Draw the half lines $L_1$ and $L_2$ from the origin through $z_1$ and $z_2$. Let $\zeta_1,\zeta_2$ be the points in which $L_1,L_2$ intersect a circle $C$ about the origin. If $C$ is described in the positive sense, the arc $C_1$ from $\zeta_1$ to $\zeta_2$ does not intersect the negative axis, and the arc $C_2$ from $\zeta_2$ to $\zeta_1$ does intersect the positive axis (since $z_1$ lies in the lower half plane while $z_2$ lies in the upper half plane). Denote the directed line segments from $z_1$ to $\zeta_1$ and from $z_2$ to $\zeta_2$ by $\delta_1$ and $\delta_2$, respectively. Introducing the closed curves $\sigma_1=\gamma_1+\delta_2-C_1-\delta_1$ and $\sigma_2=\gamma_2+\delta_1-C_2-\delta_2$, we find that $$W(\gamma, 0)=W(C_1,0)+W(C_2,0)+W(\sigma_1,0)+W(\sigma_2,0)-W(\delta_2,0)+W(\delta_1,0)+W(\delta_2,0)-W(\delta_1,0)=W(C,0)+W(\sigma_1,0)+W(\sigma_2,0).$$ But $\sigma_1$ does not meet the negative axis. Hence the origin belongs to the unbounded region determined by $\sigma_1$, and therefore $W(\sigma_1,0)=0$. Similarly, $\sigma_2$ does not meet the positive axis, so $W(\sigma_2,0)=0$. Finally, since $C$ is a circle about the origin, $W(C,0)=1$. Thus we conclude that $$W(\gamma, 0)=1+0+0=1.$$

\begin{figure}[h]
\centering
\begin{asy}
import graph;
size(10cm);

// Set up coordinate system
real xmin = -3, xmax = 3, ymin = -2.5, ymax = 2.5;

// Draw axes
draw((xmin,0)--(xmax,0), gray+0.5bp, Arrow);
draw((0,ymin)--(0,ymax), gray+0.5bp, Arrow);

// Define key points
pair z1 = (2.2, -1.5);  // Point in lower half-plane
pair z2 = (-1.8, 2.0);  // Point in upper half-plane
pair origin = (0, 0);

// Draw circle C about origin through a convenient radius
real radius = 1.0;
path circle_C = circle(origin, radius);
draw(circle_C, black+0.5bp);
label("$C$", (radius*cos(pi/6), radius*sin(pi/6)), NE);

// Draw half-lines L1 and L2 from origin through z1 and z2
real line_length = 2.5;
pair L1_end = origin + line_length * unit(z1);
pair L2_end = origin + line_length * unit(z2);

draw(origin--L1_end, dashed+gray+0.5bp);
draw(origin--L2_end, dashed+gray+0.5bp);
label("$L_1$", L1_end + (0.1, -0.1), SE);
label("$L_2$", L2_end + (-0.1, 0.1), NW);

// Find intersection points ζ₁ and ζ₂ of half-lines with circle
pair zeta1 = origin + radius * unit(z1);
pair zeta2 = origin + radius * unit(z2);

// Draw and label the intersection points
dot(zeta1, black+3bp);
dot(zeta2, black+3bp);
label("$\zeta_1$", zeta1 + (0.15, 0.15), NE);
label("$\zeta_2$", zeta2 + (-0.25, 0.15), NW);

// Draw the closed curve γ
// γ₁: from z1 to z2 (avoiding negative real axis)
path gamma1 = z1{N+E}..{W+N}(1.5, 1.2)..{W+N}(0.3, 1.8)..{W+N}z2;
draw(gamma1, blue+0.7bp, Arrow(Relative(0.7)));

// γ₂: from z2 to z1 (avoiding positive real axis)  
path gamma2 = z2{S+W}..{E+S}(-1.2, -0.8)..{E+S}(0.5, -1.5)..{E+S}z1;
draw(gamma2, red+0.7bp, Arrow(Relative(0.7)));

// Draw and label the curve points
dot(z1, black+2bp);
dot(z2, black+2bp);
label("$z_1$", z1 + (0, -0.2), S);
label("$z_2$", z2 + (0, 0.2), N);

// Label the curve segments
label("$\gamma_1$", (0.5, 2.2), blue);
label("$\gamma_2$", (-1.5, -1.8), red);
label("$\gamma$", (2.7, 0.2), black);

// Draw the line segments δ₁ and δ₂
draw(z1--zeta1, black+0.5bp+dashed);
draw(z2--zeta2, black+0.5bp+dashed);
label("$\delta_1$", (z1+zeta1)/2 + (0.2, 0), E);
label("$\delta_2$", (z2+zeta2)/2 + (-0.2, 0), W);

// Draw arcs C₁ and C₂ on the circle
// C₁: from ζ₁ to ζ₂ (not intersecting negative axis)
real angle1 = atan2(zeta1.y, zeta1.x);
real angle2 = atan2(zeta2.y, zeta2.x);

// Ensure we go the shorter way that avoids negative axis
if (angle2 < angle1) angle2 += 2*pi;
path arc_C1 = arc(origin, radius, degrees(angle1), degrees(angle2));
draw(arc_C1, heavygreen+0.7bp, Arrow(Relative(0.5)));
label("$C_1$", radius*(cos((angle1+angle2)/2), sin((angle1+angle2)/2)) + (0.2, 0.1), align=NoAlign, heavygreen);

// C₂: from ζ₂ to ζ₁ (the complementary arc, intersecting positive axis)
path arc_C2 = arc(origin, radius, degrees(angle2), degrees(angle1+2*pi));
draw(arc_C2, orange+0.7bp, Arrow(Relative(0.5)));
label("$C_2$", radius*(cos((angle1+angle2)/2 + pi), sin((angle1+angle2)/2 + pi)) + (-0.2, -0.1), align=NoAlign, orange);

// Add origin point
dot(origin, black+2bp);
label("$0$", origin + (-0.15, -0.15), SW);
\end{asy}
\end{figure}
\end{proof}

This proof is a brilliant example of when analytical proofs trump topological ones. Had we started off by defining winding numbers entirely topologically, we would have had to digress into homotopy invariance, differential forms, and Stokes' Theorem, then proven the degree formula to relate the topological and analytical definitions of winding number, and only then prove the integral formula. Along with all of this, we would have to define some more abstract notion of orientation in terms of equivalence classes of bases of tangent spaces to smooth manifolds, rather than sticking to the simple, easily visualized concepts of left and right. All of this is very doable, beautiful in its own way, and is the more general and correct course of action to perform analysis on higher-dimensional manifolds. But we do not need any of it for complex analysis. See the Appendix for a glimpse at the topological machinery.

\section{The Integral Formula}
We are now ready to state and prove Cauchy's Integral Formula.

\begin{theorem}[Cauchy's Integral Formula]
\label{thm:cauchy-integral-formula}
Suppose that $f(z)$ is analytic in an open disk $\Delta$, and let $\gamma$ be a closed curve in $\Delta$. For any point $a$ not on $\gamma$, we have $$W(\gamma, a) \cdot f(a)=\dfrac{1}{2\pi i}\int_{\gamma} \dfrac{f(z)dz}{z-a}.$$
\end{theorem}
\begin{proof}
If $a \notin \Delta$, then $W(\gamma, a)=0$ and $\int_{\gamma} \frac{f(z)dz}{z-a}=0$ since the integrand is analytic in $\Delta$, so the theorem is vacuously true regardless of what value $f$ takes on at $a$. We may therefore assume that $a \in \Delta$.

We apply Cauchy's Theorem to the function $$F(z)=\dfrac{f(z)-f(a)}{z-a}$$, which is analytc for all $z \neq a$. For $z \neq a$ it is not defined, but it satisfies the condition $$\lim_{z \rightarrow a}F(z)(z-a)=\lim_{z \rightarrow a}(f(z)-f(a))=0,$$ which is the condition of Theorem \ref{thm:cauchy-disk-stronger}. We conclude that $$\int_{\gamma} \dfrac{f(z)-f(a)}{z-a}dz=0.$$ This equation can be written in the form $$\int_{\gamma} \dfrac{f(z)dz}{z-a}=f(a)\int_{\gamma} \dfrac{dz}{z-a},$$ and we observe that the integral on the right-hand side is by definition $2\pi i \cdot W(\gamma, a)$. This completes the proof.
\end{proof}

The most common application of Cauchy's Integral Formula is the case $W(\gamma, a)=1$; we have then $$f(a)=\dfrac{1}{2\pi i}\int_{\gamma} \dfrac{f(z)dz}{z-a},$$ and this we interpret as a \emph{representation formula}. Indeed, it permits us to compute $f(a)$ as soon as the values of $f(z)$ on $\gamma$ are given, together with the fact that $f(z)$ is analytic in $\Delta$. Letting $a$ take different values, provided that its winding number with respect to $\gamma$ remains $1$, we can treat it as a parameter in and of itself, and change the notation to write $$f(z)=\dfrac{1}{2\pi i}\int_{\gamma} \dfrac{f(\zeta)d\zeta}{\zeta-z}.$$ It is this formula that is usually referred to as \emph{Cauchy's Integral Formula}. We must remember that it is valid only for points $z$ in the region determined by $\gamma$ for which $W(\gamma, z)=1$, and that we have proved it only when $f(z)$ is analytic in a disk.

Cauchy's Integral Formula forms part of a class of results that all end up, in some way or another, extrapolating information from the boundary of a domain to its interior. At the risk of sounding like a broken record, we allude yet again to topology and the classic theorems of Green and Stokes as well as the Boundary Theorem from intersection theory. 
